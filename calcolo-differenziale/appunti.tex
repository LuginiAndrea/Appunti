\section{Insiemi numerici}
    E' detta \textbf{insieme} una \textbf{collezione} di elementi per i quali è 
    sempre possibile rispondere alla domanda $x \in A$.
    \subsection{Applicazioni}
    Tramite un'applicazione, associo gli elementi dell'insieme A
    agli elementi dell'insieme B, detto \textbf{immagine} di A. \\
    Un'applicazione è:
    \begin{itemize}
        \item \textbf{Iniettiva}: $
            \forall x_1, x_2 \in A \textrm{t.c.} x_1 \neq x_2:
            x_1 \rightarrow b_1 \neq x_2 \rightarrow b_2
        $
        \item \textbf{Suriettiva}: $\forall b \in B: 
            \exists a \in A \textrm{t.c.} a \rightarrow b
        $
        \item \textbf{Biunivoca}: Suriettiva $\wedge$ Iniettiva
    \end{itemize}
    Se esiste un'applicazione suriettiva ed iniettiva fra $A$ e $B$ questi sono
    detti in \textbf{biezione}.
    \subsection{Definizione di $\mathbf{N}$ tramite gli insiemi}
    A questo punto possiamo definire i numeri naturali positivi partendo dagli insiemi.
    \begin{itemize}
        \item \textbf{0}: classe degli insiemi in biezione con $A = \emptyset$
        \item \textbf{1}: classe degli insiemi in biezione con $A = {\emptyset}$
        \item \textbf{2}: Classe degli insiemi in biezione con $A = {\emptyset, {\emptyset}}$
        \item \textbf{...}: ...
    \end{itemize}
    Possiamo accorgerci quindi come l'insieme $\mathbf{N}$ definisce la \textbf{cardinalità}
    degli insiemi. \\
    Da qui possiamo continuare:
    \begin{itemize}
        \item \textbf{$\mathbf{N} +$}: poichè $\mathbf{N}$ definisce la cardinalità degli insiemi,
            la somma di due numeri $\in \mathbf{N}$ è uguale alla cardinalità di 
            $(A \cup B) \forall A, B \textrm{t.c.} A \cap B = \emptyset$.
        \item \textbf{-1}: quel numero t.c. $-1 + 1 = 0$. Da qui definiamo $\mathbf{Z}$.
        \item \textbf{$\frac{n}{m}$}: 
            $a_1, a_2, ..., a_n \in \mathbf{Z}, 0 \leq a_2, ..., a_n \leq 9:
            a_1 + \Sigma_{i=1}{n}\frac{a^i}{10^i}$
        \item \textbf{$\mathbf{Q}$}: 
            ${\frac{n}{m},\frac{n_1}{m_1} \in \mathbf{Z}, m, m_1 \neq 0: 
            (n, m) = (n_1, m_1) \Longleftrightarrow (n * m_1) = (n_1 * m)}$. \\
            La struttura \textbf{periodica} è valida se si conviene che 
            $a,\overline{9} = a + 1$.
        \item \textbf{$\mathbf{Q} +$}: 
            $\frac{n}{m} + \frac{n_1}{m_1} = \frac{n * m_1 + m * n_1}{m * m_1}$
        \item \textbf{$\mathbf{Q} *$}:
            $\frac{n}{m} * \frac{n_1}{m_1} = \frac{n * n_1}{m * m_1}$
        \item \textbf{$\mathbf{Q}$ inverso}: 
            $\frac{n}{m} * \frac{n}{m}^-1 = 1 \Longrightarrow 
            \frac{n}{m}^-1 = \frac{m}{n}, n \neq 0$
        \item \textbf{$\mathbf{R}$}: {Tutti i numeri scritti in forma decimale
            anche con \textbf{infinite} cifre \textbf{non periodiche} dopo la virgola}
    \end{itemize}
    Possiamo infine definire le n-tuple di numeri $\left(a, b, ...\right)$ come
    \textbf{prodotto cartesiano} degli insiemi $A * B * ... = \{(a, b, ...) 
    \forall a \in A \wedge \forall b \in B \wedge ...\}$
    $$\mathbf{N} \subset \mathbf{Z} \subset \mathbf{Q} \subset \mathbf{R}$$
\section{Coefficiente binomiale}
    $$q \in R, q \neq 1 \Longrightarrow \Sigma_{k=0}^{n}q^k = \frac{1-q^{n+1}}{1-q} \Longleftrightarrow$$
    $$(1-q) * \Sigma_{k=0}^{n}q^k = 1-q^{n+1} \Longleftrightarrow$$
    $$\Sigma_{k=0}^{n}q^k - q * \Sigma{k=0}^{n}q^k = 1-q^{n+1} \Longleftrightarrow$$
    $$\Sigma_{k=0}^{n}q^k - \Sigma{k=1}^{n+1}q^k = 1 - q^{n+1} \Longleftrightarrow$$
    $$\left(\Sigma_{k=1}^{n}q^k + 1\right) - \left(\Sigma_{k=1}^{n}q^k + q^{n+1}\right) = 1 - q^{n+1} \Longleftrightarrow$$
    $$\Sigma{k=1}^{n}q^k - \Sigma{k=1}^{n}q^k + 1 - q^{n+1} = 1 - q^{n+1} \Longleftrightarrow$$
    $$\Sigma{k=1}^{n}q^k = \Sigma{k=1}^{n}q^k$$ 
    \\ \\
    $$\binom{n}{k} = \binom{n-1}{k-1} + \binom{n-1}{k} \Longleftrightarrow$$
    $$\frac{n!}{k!\left(n-k\right)!} = 
        \frac{\left(n-1\right)!}{\left(k-1\right)!\left(n-1-\left(k-1\right)\right)!} + \frac{\left(n-1\right)!}{k!\left(n-1-k\right)!} \Longleftrightarrow$$
    $$\frac{n!}{k!\left(n-k\right)!} = 
        \frac{\left(n-1\right)!}{\left(k-1\right)!\left(n-k\right)\left(n-k-1\right)!} + \frac{\left(n-1\right)!}{k\left(k-1\right)!\left(n-1-k\right)!} \Longleftrightarrow$$
    $$\frac{n!}{k!\left(n-k\right)!} = 
        \frac{k\left(n-1\right)! + \left(n-k\right)\left(n-1\right)!}{k!\left(n-k\right)!} \Longleftrightarrow$$
    $$\frac{n!}{k!\left(n-k\right)!} = 
        \frac{\left(k+n-k\right)!}{k!\left(n-k\right)!} \Longleftrightarrow$$
    $$\frac{n!}{k!\left(n-k\right)!} = \frac{n * \left(n-1\right)!}{k!\left(n-k\right)!} \Longleftrightarrow$$
    $$\frac{n!}{k!\left(n-k\right)!} = \frac{n!}{k!\left(n-k\right)!}$$
    \\ \\
    $$\left(a+b\right)^n = \Sigma_{k=0}^{n}\left(\binom{n}{k} * a^{n-k} * b^k\right)$$
    Dimostriamolo per induzione usando il seguente schema.
    \begin{enumerate}
        \item $P\left(n\right)$ è vera con $n=1$
        \item Supponiamo che $P\left(n\right)$ vera $\Longrightarrow P\left(n+1\right)$ vera
    \end{enumerate}
    Procediamo al primo passo:
    $$\left(a+b\right)^1 = \Sigma_{k=0}{1}\left(\binom{1}{k} * a^{1-k} * b^k\right) \Longleftrightarrow$$
    $$a+b = \binom{1}{0} * a^1 * b^0 + \binom{1}{1} * a^{1-1} * b^1 \Longleftrightarrow$$
    $$a+b = 1 * a * 1 + 1 * 1 * b = a + b$$
    Abbiamo dimostrato che $P\left(1\right)$ è vera, procediamo quindi col secondo passaggio.
    $$\left(a+b\right)^{n+1} = \Sigma_{k=0}^{n+1}\left(\binom{n+1}{k} * a^{n+1-k} * b^k\right)$$
    $$\left(a+b\right)\left(a+b\right)^n = $$ 
    $$\left(a+b\right) * \Sigma_{k=0}^{n}\left(\binom{n}{k} * a^{n-k} * b^k\right) = $$ xù
    $$\Sigma_{k=0}^{n}\left(\binom{n}{k} * a^{n+1-k} * b^k\right) + 
        \Sigma_{k=0}^{n}\left(binom{n}{k} * a^{n-k} * b^{k+1}\right) = $$ 
    $$\Sigma_{k=0}^{n}\left(\binom{n}{k} * a^{n+1-k} * b^k\right) +  
        \Sigma_{k=1}^{n+1}\left(\binom{n}{k-1} * a^{n-k+1} * b^k\right) = $$ 
    $$\binom{n}{0} * a^{n+1} * b^0 + \Sigma_{k=1}^{n}\left(\binom{n}{k} * a^{n+1-k} * b^k\right) +
        \Sigma_{k=1}^{n}\left(\binom{n}{k-1} * a^{n-k+1} * b^k\right) + \binom{n}{n} * a^0 * b^{n+1} = $$
    $$a^{n+1} + 
        \Sigma_{k=1}^{n}\[\left(\binom{n}{k} + \binom{n}{k-1}\right) * a^{n+1-k} * b^k\]
        b^{n+1} = $$ 
    $$a^{n+1} + \Sigma_{k=1}^{n}\left(\binom{n+1}{k} * a^{n+1-k} * b^k\right) + b^{n+1} = $$
    $$\Sigma_{k=0}^{n+1}\left(binom{n+1}{k} * a^{n+1-k} * b^{k}\right)$$
\section{Funzioni}
    \subsection{Definizione}
        Dati due insiemi $A$ e $B$, una funzione con \textbf{dominio} $A$ e 
        \textbf{codominio} B è una qualunque legge che \textbf{ad ogni} elemento
        di $A$ associa \textbf{uno ed uno solo} elemento di $B$.\\
        Può anche essere ad \textbf{n variabili} ed avere quindi n insiemi di partenza
        $$f: A \longrightarrow B \textrm{t.c. } \forall x \in A \longrightarrow f(x) \in B$$
        Le funzioni \textbf{reali} a variabile \textbf{reale} sono le funzioni
        $$f: A \subset \mathbf{R} \longrightarrow \mathbf{R}$$
    \subsection{Immagine}
        $$ \{f(x) \forall x \in A\} \subset B $$
    \subsection{Grafico di una funzione}
        \subsubsection{Definizione}
            L'insieme dei punti $\mathbf{R}^2$ definiti da 
            $$ g_{\mathbf{R}} = \{(x,f(x)) \forall x \in A\}$$
        \subsubsection{Rappresentazione sul piano}
            Poichè $\mathbf{R}^2$ è rappresentabile sul piano cartesiano anche 
            $g_{\mathbf{R}}$ lo è
        \subsubsection{Proprietà fondamentale della funzione espressa col grafico}
        $$\forall x_0 \in A \exists!y_0 \textrm{t.c. } {x = x_0} \cap {g_{\mathbf{R}}}
            = {(x_0, f(x_0))}$$
    \subsection{Proprietà delle funzioni}
        Una funzione è detta
        \begin{itemize}
            \item \textbf{pari}: $\forall x \in A: f(x) = f(-x)$
            \item \textbf{dispari}: $\forall x \in A: -f(x) = f(-x)$
            \item \textbf{limitata superiormente}: $\exists M \in \mathbf{R} 
                \textrm{t.c. } M \geq f(x) \forall x \in A$
            \item \textbf{limitata inferiormente}: $\exists m \in \mathbf{R} 
                \textrm{t.c. } m \leq f(x) \forall x \in A$
            \item \textbf{limitata}: $f(x)$ è limitata superiormente e inferiormente
            \item \textbf{monotona crescente in A}: $\forall x_1, x_2 \in A \textrm{t.c. } 
                x_1 \leq x_2: f(x_1) \leq f(x_2)$
            \item \textbf{monotona decrescente in A}: $\forall x_1, x_2 \in A \textrm{t.c. } 
                x_1 \leq x_2: f(x_1) \geq f(x_2)$
            \item \textbf{periodica di periodo T}: $\forall x \in A, x + kT \in A, 
                k \in \mathbf{Z}: f(x + kT) = f(x)$
        \end{itemize}
    % \subsection{Funzioni elementari}
    % \begin{itemize}
