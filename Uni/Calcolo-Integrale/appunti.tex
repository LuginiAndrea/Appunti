\documentclass{article}
\usepackage{amssymb}
\usepackage{amsmath}
\newtheorem{theorem}{Theorem}[section]
\newtheorem{corollary}{Corollary}[theorem]
\newtheorem{lemma}{Lemma}[theorem]
\newtheorem{definition}{Definition}[section]
% \newcommand{\implies}{\Rightarrow}
\newcommand{\qed}{\rule{0.7em}{0.7em}}
\begin{document}
\title{Calcolo Integrale}
\maketitle
\section{Serie}
    \subsection{Definizione}
        Una serie $\{S_n\}$ della serie $\{a_n\}$ è la \textbf{successione delle somme parziali} 
        di $a_n$, ovvero 
        $$\forall n \in \mathbb{N}, S_n = \sum_{k=0}^n a_k$$
    \subsection{Carattere di una serie}
        Una serie, per $n \to \infty$, può fare 3 cose, esattamente come una successione può:
        \begin{itemize}
            \item \textbf{Divergere}: $\sum_{k=0}^\infty a_k = \pm \infty$
            \item \textbf{Convergere}: $\sum_{k=0}^\infty a_k \in \mathbb{R}$
            \item \textbf{Non convergere}: $\nexists \sum_{k=0}^\infty a_k$
        \end{itemize}
    \subsection{Condizione necessaria per la convergenza di una serie}
        \begin{theorem}[Condizione necessaria per la convergenza della serie]
            $$\sum_{k=0}^\infty a_k \in \mathbb{R} \implies a_k \to 0$$
        \end{theorem}
        La dimostrazione è alquanto semplice: \\
        Se $S_n$ converge per $n \to \infty$, allora 
        $$\lim_{n \to \infty} S_n = S_{n+1} = S \in \mathbb{R} \implies 
            \lim_{n \to \infty} \sum_{k=0}^{n+1} a_k - \lim_{n \to \infty} \sum_{k=0}^{n} a_k = 
            \lim_{n \to \infty} a_{n+1} = 0
        $$
        Una riformulazione equivalente è la seguente
        \begin{theorem}[Criterio di Cauchy per la convergenza di una serie] 
            $$\{S_n\} \textrm{ converge } \iff \forall m > n, |S_m - S_n| = 0 \textrm{ definitivamente }$$
        \end{theorem}
        \textit{Dimostrazione}: \\
            Se $\{S_n\}$ converge ad $l$, allora $|\lim_{n \to \infty} S_n - \lim_{m \to \infty} S_m| = l - l = 0$. \\
            Ipotizziamo ora che $|S_m - S_n| \neq 0$ definitivamente. Questo implica che, 
            $$\forall n \in \mathbb{N}, \exists m > n \in \mathbb{N}, \exists \Delta  \in \mathbb{R}
                \textrm{ t.c. } |S_m - S_n| = \Delta  = \sum_{i=n+1}^m a_i \implies a_m \neq 0$$
            Per il criterio di convergenza di una serie ne consegue che $\{S_n\}$ non converge.
        \subsubsection{Nota bene}
            Il criterio non implica che se $a_n \to 0$ allora $S_n$ converge. Un esempio è la serie 
            $$\sum_{k=1}^\infty k^{-1} = \infty$$
    \subsection{Serie a termini costanti}
        \begin{theorem}[Carattere di serie a termini costanti]
            $$a_n \geq/\leq 0 \textrm{ definitivamente } \implies \lim_{n \to \infty} S_n \in \mathbb{R} \, \lor = \pm \infty$$
        \end{theorem}
        Questa proprietà è una semplice conseguenza delle successioni monotone
        \begin{corollary}
            $$\{a_k\} \textrm{ definitivamente di segno costante } \land a_k \not\to 0 \implies \{S_n\} \textrm{ diverge }$$
        \end{corollary}
        \begin{corollary}
            $$\{S_n\} \textrm{ limitata dall'alto/basso } \land \{a_k\} \textrm{ definitivamente } \geq/\leq 0 \implies \{S_n\} \textrm{ converge}$$
        \end{corollary}
    \subsection{Serie geometriche}
        $$\{S_n\} \textrm{ è una serie geometrica } \iff \forall k \in \mathbb{N} a_k = q^k \land q \in \mathbb{R}$$
        \begin{theorem}[Carattere della serie geomtrica]
            Una serie geometrica ha carattere diverso in base al valore di $q$, detto ragione della serie:
            \begin{itemize}
                \item $q = 0$: $\forall n \in \mathbb{N}, S_n = 0$. Converge a 0 per $n \to \infty$
                \item $q = 1$: $S_n = n + 1$. Diverge a $\infty$ per $n \to \infty$
                \item $q = -1$: Se $2 | n$, $S_n = 1$, in alternativa $S_n = 0$. Non converge per $n \to \infty$
                \item Per i restanti $q$, $S_n = \frac{1-q^{n+1}}{1-q}$. 
            \end{itemize}
        \end{theorem}
        \subsubsection{Dimostrazione della formula per il calcolo dell'$n$-esimo valore della serie geomtrica}
            $$S_n = \sum_{k=0}^n q^k \implies q \cdot S_n = \sum_{k=0}^n q^{k+1} \implies 
                S_n - qS_n = 1 - q^{k+1} = S_n\left(1 - q\right) \implies$$
            $$S_n = \frac{1 - q^{n+1}}{1 - q}$$
    \subsection{Serie armoniche}
        $$\{S_n\} \textrm{ è una serie armonica } \iff \forall k \in \mathbb{N}, \, a_k = k^\alpha, \alpha \in \mathbb{R}$$
        \begin{theorem}[Convergenza della serie armonica generalizzata]
                $$\{S_n\} \textrm{ converge } \iff \alpha > 1$$
        \end{theorem}
        La dimostrazione usa il criterio dell'integrale (vedremo la dimostrazione quando parlaremo del criterio)
    \subsection{Criteri per studiare il carattere di una serie}
        \begin{theorem}[Criterio del confronto]
            $$\forall \{A_n\}, \{B_n\} \textrm{ di termini a segno definitivamente costante }, a_n \leq b_n \textrm{ definitivamente } \implies$$
            $$\left(A_n \to \infty \implies B_n \to \infty\right) \land 
                \left(B_n \to l_B \in \mathbb{R} \implies A_n \to l_A \in \mathbb{R}\right)$$
        \end{theorem}
        La dimostrazione è alquanto ovvia e la lascio per esercizio al lettore
        \begin{theorem}[Criterio del confronto asintotico]
            $$\forall \{A_n\}, \{B_n\} \textrm{ di termini a segno definitivamente costante}, \lim_{n \to \infty }\frac{a_n}{b_n} = L$$
            \begin{equation}
                \left\{\begin{array}{l l l l}
                    A_n \to +\infty & \iff & B_n \to \infty & L \in \mathbb{R}^+ \\
                    B_n \to \infty & \implies & A_n \to \infty & L = \infty \\
                    B_n \to l_B \in \mathbb{R} & \implies & A_n \to l_A \in \mathbb{R} & L = 0 \\
                \end{array}\right.
            \end{equation}
        \end{theorem}
        Anche qui la dimostrazione è simile a quella del confronto e la lascio al lettore
        \begin{theorem}[Criterio del rapporto]
            $$\{A_n\} \textrm{ di termini a segno definitivamente costante }, \lim_{n \to \infty}\frac{a_{n+1}}{a_n} = L$$
            \begin{equation}
                \left\{\begin{array}{l l}
                    A_n \to +\infty & L > 1 \\
                    A_n \to l_A \in \mathbb{R} & L < 1 \\
                    \textrm{Inconcludente} & L = 1 \\
                \end{array}\right.
            \end{equation}
        \end{theorem}
        Dimostrazione: se il limite per $n \to \infty$ del rapporto tende a $L < 1$, 
        per la completezza di $\mathbb{R}$ esistono $N \in \mathbb{N}, r \in \left[\left.L, 1\right)\right.$
        tale che $a_{n+1} < ra_n \, \forall n > N$, che iterativamente implica che 
        $$\forall i \in \mathbb{N}, a_{n+i} < r^ia_n \implies \sum_{i=1}^\infty a_{N+i} < \sum_{i=1}^\infty r^ia_n$$
        Essendo la seconda una serie geometrica con $0 < r < 1$, questa converge, e per il criterio del confronto 
        $\{A_n\}$ converge
        \begin{theorem}[Criterio della radice]
            $$\{A_n\} \textrm{ di termini a segno definitivamente costante }, \lim_{n \to \infty}\sqrt[n]{a_n} = L$$
            \begin{equation}
                \left\{\begin{array}{l l}
                    A_n \to +\infty & L > 1 \\
                    A_n \to l_A \in \mathbb{R} & L < 1 \\
                    \textrm{Inconcludente} & L = 1 \\
                \end{array}\right.
            \end{equation}
        \end{theorem}
        La dimostrazione è simile a quella di prima: se $L < 1$ allora 
        esistono $N \in \mathbb{N}, r \in \left[\left.L, 1\right)\right.$ tale che 
        $\forall n > N, L^n = a_n < r^n < 1$, quindi 
        $$\sum_{n = N}^\infty a_n < \sum_{n=N}^\infty r^n$$ Essendo la 
        seconda una serie geomtrica convergente, per il teorema del confronto $\{A_n\}$ converge
    \begin{theorem}[Criterio di Leibniz]
        $$\{A_n\} \textrm{ con } A_n = \sum_{k=0}^n \left(-1\right)^ka_k, \{a_n\} \textrm{ definitivamente monotona decrescente e di segno costante }$$
        $$\implies \{A_n\} \to l_A \in \mathbb{R}$$
    \end{theorem}
    \textit{Dimostrazione}:
        In quanto la serie è monotona decrescente, ne segue che $|a_n - a_{n+1}| \leq |a_n| \forall n$ \\
        Da questa idea ne segue facilmente (dimostrabile per induzione) che vale la disugaglianza
        $$|a_m - \sum_{i = n+1}^{\infty} a_i| \leq |a_m - a_{m+1}| \leq |a_m| \forall m$$
        Ma in quanto $lim_{n \to \infty} a_n = 0$ abbiamo che 
        $$\forall m > n > N \in \mathbb{N}, \lim_{n \to \infty} |S_m - S_n| = |a_m - \sum_{i=n+1}^\infty \pm a_i| \leq a_m = 0$$
        Ne consegue per il criterio di Cauchy che la serie converge.
    \begin{theorem}[Criterio della convergenza assoluta]
        $$\sum_{i=0}^\infty |a_n| \textrm{ converge } \implies \sum_{i=0}^\infty a_n \textrm{ converge }$$
    \end{theorem}
\section{Formule importanti}
    \begin{definition}[Formula di approssimazione di Stirling]
        $$n! \sim \sqrt{2n\pi}\left(\frac{n}{e}\right)^n$$
    \end{definition}
    \begin{definition}[$e^x$ come serie]
        $$e^x = \sum_{k=0}^\infty \frac{x^k}{k!}$$
    \end{definition}
    \begin{definition}[Comportamento asintotico del logaritmo del fattoriale]
        $$\lim_{x \to \infty}\frac{\ln{x!}}{x} = \infty$$
    \end{definition}
    \textit{Dimostrazione}:
        Sappiamo che, per $x \to \infty$, $x! > \alpha^x \forall \alpha$ Consideriamo allora 
        tutti gli $\alpha$ nella forma $$\alpha = e^{k}, k \in \mathbb{N}$$
        Consideriamo ora $$\lim_{x \to \infty}\frac{\ln{e^{kx}}}{x} = \frac{kx}{x} = {k}$$
        Poichè ciò vale $\forall k$, vale per $k$ grande a piacere, di conseguenza per $k \to \infty$, 
        il che implica che 
        $$\lim_{x \to \infty}\frac{\ln{x!}}{x} > \lim_{x \to \infty}\frac{\ln{e^{kx}}}{x} \to \infty \; \qed$$
    \begin{definition}[$e$ come limite]
        $$e = \lim_{x \to \infty}\frac{x}{\sqrt[x]{x!}}$$
    \end{definition}
    \textit{Dimostrazione}:
        Usando l'approssimazione di Stirling otteniamo
        $$\lim_{x \to \infty}\frac{x}{\sqrt[2x]{2x\pi}} \times \left(\frac{e}{x}\right)^{x \times 1/x} = 
            \lim_{x \to \infty}\frac{x}{\sqrt[2x]{2x\pi}} \times \frac{e}{x} = 
            \lim_{x \to \infty}\frac{e}{\sqrt[x]{2x\pi}^{1/2}}$$
        Recall that for $x \to \infty, x^{1/x} \sim 1$, we get for $x \to \infty$ that our 
        limit $\sim e/1 = e \; \qed$
    Ricordarsi bene anche limiti notevoli ed espansioni di Taylor-MacLaurin
    
            

\end{document}